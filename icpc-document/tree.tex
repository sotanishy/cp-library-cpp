\section{tree}

\subsection{Lowest Common Ancestor}

\begin{small}
\begin{markdown}
- `LCA(vector<vector<int>> G, int root)`
    - 前計算をする
    - 時間計算量: $O(n\log n)$
- `int query(int u, int v)`
    - 頂点 $u$ と頂点 $v$ の最小共通祖先を返す
    - 時間計算量: $O(\log n)$
- `int dist(int u, int v)`
    - $uv$ 間の距離を計算する
    - 時間計算量: $O(\log n)$
- `int parent(int v, int k)`
    - 頂点 $v$ の $k$ 個上の頂点を求める
- `int jump(int u, int v, int k)`
    - $uv$ パス上の $k$ 番目の頂点を返す.$k=0$ のとき $u$ を,$k>dist(u,v)$ のとき $-1$ を返す.
    - 時間計算量: $O(\log n)$
\end{markdown}
\end{small}

\begin{lstlisting}
class LCA {
public:
    LCA() = default;
    LCA(const std::vector<std::vector<int>>& G, int root) : G(G), LOG(32 - __builtin_clz(G.size())), depth(G.size()) {
        int V = G.size();
        table.assign(LOG, std::vector<int>(V, -1));

        dfs(root, -1, 0);

        for (int k = 0; k < LOG - 1; ++k) {
            for (int v = 0; v < V; ++v) {
                if (table[k][v] >= 0) {
                    table[k + 1][v] = table[k][table[k][v]];
                }
            }
        }
    }

    int query(int u, int v) const {
        if (depth[u] > depth[v]) std::swap(u, v);

        // go up to the same depth
        for (int k = 0; k < LOG; ++k) {
            if ((depth[v] - depth[u]) >> k & 1) {
                v = table[k][v];
            }
        }
        if (u == v) return u;

        for (int k = LOG - 1; k >= 0; --k) {
            if (table[k][u] != table[k][v]) {
                u = table[k][u];
                v = table[k][v];
            }
        }
        return table[0][u];
    }

    int dist(int u, int v) const {
        return depth[u] + depth[v] - 2 * depth[query(u, v)];
    }

    int parent(int v, int k) const {
        for (int i = LOG - 1; i >= 0; --i) {
            if (k >= (1 << i)) {
                v = table[i][v];
                k -= 1 << i;
            }
        }
        return v;
    }

    int jump(int u, int v, int k) const {
        int l = query(u, v);
        int du = depth[u] - depth[l];
        int dv = depth[v] - depth[l];
        if (du + dv < k) return -1;
        if (k < du) return parent(u, k);
        return parent(v, du + dv - k);
    }


protected:
    const std::vector<std::vector<int>>& G;
    const int LOG;
    std::vector<std::vector<int>> table;
    std::vector<int> depth;

    void dfs(int v, int p, int d) {
        table[0][v] = p;
        depth[v] = d;
        for (int c : G[v]) {
            if (c != p) dfs(c, v, d + 1);
        }
    }
};
\end{lstlisting}

\subsection{Tree Isomorphism}

\begin{small}
\begin{markdown}
やり方だけ説明する

- ハッシュ
    - 確率的 $O(n)$
    - 部分木 $v$ の深さを $d$ とし,$d$ に対応する乱数 $R\_d$ を取る.$v$ の子のハッシュを $h\_1,\dots,h\_k$ として,$v$ のハッシュ $h$ を $h=\prod (h\_i+R)$ とする.
- AHU algorithm
    - 決定的 $O(n\log n)$
    - 子のラベルの列をソートして,その列に対して新しいラベルを付与することを繰り返す
    - 実装が楽だし決定的で嬉しいのでTLが厳しくないならこっちが良さそう

根付き木でない場合は,中心を根にしてやればよい
\end{markdown}
\end{small}

\subsection{Centroid Decomposition}

\begin{small}
\begin{markdown}
- `tuple<vector<int>, vector<int>, vector<int>> centroid\_decomposition(vector<vector<int>> G)`
  - 木 $G$ の隣接リストが与えられたとき,3つ組 `(level, sz, par)` を返す.
    - `level`: $G$ を重心分解したときの各頂点のレベル (何回目の分割でそれが重心となるか)
    - `sz`: 各頂点が重心となるときにそれが含まれる部分木のサイズ
    - `par`: 各頂点が重心となる直前に属していた部分木の重心
  - 時間計算量: $O(n\log n)$

\end{markdown}
\end{small}

\begin{lstlisting}
std::tuple<std::vector<int>, std::vector<int>, std::vector<int>> centroid_decomposition(const std::vector<std::vector<int>>& G) {
    int N = G.size();
    std::vector<int> sz(N), level(N, -1), sz_comp(N), par(N);

    auto dfs_size = [&](auto& dfs_size, int v, int p) -> int {
        sz[v] = 1;
        for (int c : G[v]) {
            if (c != p && level[c] == -1) sz[v] += dfs_size(dfs_size, c, v);
        }
        return sz[v];
    };

    auto dfs_centroid = [&](auto& dfs_centroid, int v, int p, int n) -> int {
        for (int c : G[v]) {
            if (c != p && level[c] == -1 && sz[c] > n / 2) return dfs_centroid(dfs_centroid, c, v, n);
        }
        return v;
    };

    auto decompose = [&](auto& decompose, int v, int k, int p) -> void {
        int n = dfs_size(dfs_size, v, -1);
        int s = dfs_centroid(dfs_centroid, v, -1, n);
        level[s] = k;
        sz_comp[s] = n;
        par[s] = p;
        for (int c : G[s]) {
            if (level[c] == -1) decompose(decompose, c, k + 1, s);
        }
    };

    decompose(decompose, 0, 0, -1);
    return {level, sz_comp, par};
}
\end{lstlisting}

\subsection{Heavy-Light Decomposition}

\begin{small}
\begin{markdown}
木上のパスクエリ

`update` および `fold` の時間計算量を $f(n)$ とする

- `HLD(vector<vector<int>> G, bool edge)`
    - 木 `G` を HL 分解する.`edge == true` ならクエリは辺に対して実行される.
    - 時間計算量: $O(n)$
- `void update(int v, T x, F update)`
    - 頂点 $v$ に対して `update(x)` を実行する
    - 時間計算量: $O(f(n) \log n)$
- `void update(int u, int v, T x, F update)`
    - $uv$ パス上の頂点/辺に対して `update(x)` を実行する.
    - 時間計算量: $O(f(n) \log n)$
- `T path\_fold(int u, int v, F fold)`
    - $uv$ パス上の頂点/辺に対して `fold()` を実行する.
    - 時間計算量: $O(f(n) \log n)$
- `T path\_fold(int u, int v, F fold, Flip flip)`
    - $uv$ パス上の頂点/辺に対して `fold()` を実行する.値が非可換なら,左から積をとったときと右から積をとったときの値を入れ替える `flip` 関数を与える必要がある.
    - 時間計算量: $O(f(n) \log n)$
- `T subtree\_fold(int v, F fold)`
    - 頂点 $v$ を根とする部分木の頂点/辺に対して `fold()` を実行する.
    - 時間計算量: $O(f(n))$
- `int lca(int u, int v)`
    - 頂点 $u$ と頂点 $v$ の最小共通祖先を返す
    - 時間計算量: $O(\log n)$
- `int dist(int u, int v)`
    - $uv$ 間の距離を計算する
    - 時間計算量: $O(\log n)$

\end{markdown}
\end{small}

\begin{lstlisting}
template <typename M>
class HLD {
    using T = typename M::T;

public:
    HLD() = default;
    HLD(const std::vector<std::vector<int>>& G, bool edge)
        : G(G), size(G.size()), depth(G.size()), par(G.size(), -1),
          in(G.size()), out(G.size()), head(G.size()), heavy(G.size(), -1), edge(edge) {
        dfs(0);
        decompose(0, 0);
    }

    template <typename F>
    void update(int v, const T& x, const F& f) const {
        f(in[v], x);
    }

    template <typename F>
    void update_edge(int u, int v, const T& x, const F& f) const {
        if (in[u] > in[v]) std::swap(u, v);
        f(in[v], x);
    }

    template <typename E, typename F>
    void update(int u, int v, const E& x, const F& f) const {
        while (head[u] != head[v]) {
            if (in[head[u]] > in[head[v]]) std::swap(u, v);
            f(in[head[v]], in[v] + 1, x);
            v = par[head[v]];
        }
        if (in[u] > in[v]) std::swap(u, v);
        f(in[u] + edge, in[v] + 1, x);
    }

    template <typename F, typename Flip>
    T path_fold(int u, int v, const F& f, const Flip& flip) const {
        bool flipped = false;
        T resu = M::id(), resv = M::id();
        while (head[u] != head[v]) {
            if (in[head[u]] > in[head[v]]) {
                std::swap(u, v);
                std::swap(resu, resv);
                flipped ^= true;
            }
            T val = f(in[head[v]], in[v] + 1);
            resv = M::op(val, resv);
            v = par[head[v]];
        }
        if (in[u] > in[v]) {
            std::swap(u, v);
            std::swap(resu, resv);
            flipped ^= true;
        }
        T val = f(in[u] + edge, in[v] + 1);
        resv = M::op(val, resv);
        resv = M::op(flip(resu), resv);
        if (flipped) {
            resv = flip(resv);
        }
        return resv;
    }

    template <typename F>
    T path_fold(int u, int v, const F& f) const {
        path_fold(u, v, f, [&](auto& v) { return v; });
    }

    template <typename F>
    T subtree_fold(int v, const F& f) const {
        return f(in[v] + edge, out[v]);
    }

    int lca(int u, int v) const {
        while (true) {
            if (in[u] > in[v]) std::swap(u, v);
            if (head[u] == head[v]) return u;
            v = par[head[v]];
        }
    }

    int dist(int u, int v) const {
        return depth[u] + depth[v] - 2 * depth[lca(u, v)];
    }

private:
    std::vector<std::vector<int>> G;
    std::vector<int> size, depth, par, in, out, head, heavy;
    bool edge;
    int cur_pos = 0;

    void dfs(int v) {
        size[v] = 1;
        int max_size = 0;
        for (int c : G[v]) {
            if (c == par[v]) continue;
            par[c] = v;
            depth[c] = depth[v] + 1;
            dfs(c);
            size[v] += size[c];
            if (size[c] > max_size) {
                max_size = size[c];
                heavy[v] = c;
            }
        }
    }

    void decompose(int v, int h) {
        head[v] = h;
        in[v] = cur_pos++;
        if (heavy[v] != -1) decompose(heavy[v], h);
        for (int c : G[v]) {
            if (c != par[v] && c != heavy[v]) decompose(c, c);
        }
        out[v] = cur_pos;
    }
};

\end{lstlisting}

\subsection{Rerooting}

\begin{small}
\begin{markdown}
全方位木DP

DPは $dp\_v = g(f(dp\_{c\_1}, e\_1) * \dots * f(dp\_{c\_k}, e\_k), v)$ という形の遷移で表されるとする.

**Template Parameters**

- `M`
    - 可換モノイド
- `Cost`
    - 辺のコストの型
- `T apply\_edge(T a, int s, int t, Cost c)`
    - 遷移の $f$
- `T apply\_vertex(T x, int v)`
    - 遷移の $g$

**Operations**

- `Rerooting(int n)`
    - 頂点数 $n$ で木を初期化する
    - 時間計算量: $O(n)$
- `void add\_edge(int u, int v, Cost c)`
    - 頂点 $uv$ 間にコスト $c$ の辺を張る
    - 時間計算量: $O(1)$
- `vector<T> run()`
    - 各頂点を根としたときの木DPの値を求める
    - 時間計算量: $O(n)$

\end{markdown}
\end{small}

\begin{lstlisting}
template <typename M,
          typename Cost,
          typename M::T (*apply_edge)(typename M::T, int, int, Cost),
          typename M::T (*apply_vertex)(typename M::T, int)>
class Rerooting {
    using T = typename M::T;

public:
    explicit Rerooting(int n) : G(n) {}

    void add_edge(int u, int v, Cost c) {
        G[u].emplace_back(v, c);
        G[v].emplace_back(u, c);
    }

    std::vector<T> run() {
        dp_sub.resize(G.size(), M::id());
        dp_all.resize(G.size());
        dfs_sub(0, -1);
        dfs_all(0, -1, M::id());
        return dp_all;
    }

private:
    std::vector<std::vector<std::pair<int, Cost>>> G;
    std::vector<T> dp_sub, dp_all;

    void dfs_sub(int v, int p) {
        for (auto [c, cost] : G[v]) {
            if (c == p) continue;
            dfs_sub(c, v);
            dp_sub[v] = M::op(dp_sub[v], apply_edge(dp_sub[c], v, c, cost));
        }
        dp_sub[v] = apply_vertex(dp_sub[v], v);
    }

    void dfs_all(int v, int p, const T& val) {
        std::vector<T> ds = {val};
        for (auto [c, cost] : G[v]) {
            if (c == p) continue;
            ds.push_back(apply_edge(dp_sub[c], v, c, cost));
        }
        int n = ds.size();
        std::vector<T> head(n + 1, M::id()), tail(n + 1, M::id());
        for (int i = 0; i < n; ++i) head[i+1] = M::op(head[i], ds[i]);
        for (int i = n - 1; i >= 0; --i) tail[i] = M::op(ds[i], tail[i+1]);
        dp_all[v] = apply_vertex(head[n], v);
        int k = 1;
        for (auto [c, cost] : G[v]) {
            if (c == p) continue;
            dfs_all(c, v, apply_edge(apply_vertex(M::op(head[k], tail[k+1]), v), c, v, cost));
            ++k;
        }
    }
};

\end{lstlisting}

\subsection{Link/Cut Tree}

\begin{small}
\begin{markdown}
森の辺の追加,削除,根の変更,頂点の値の更新,パスクエリ

- `LinkCutTree(int n)`
    - 頂点数 $n$ で初期化する
    - 時間計算量: $O(n)$
- `void link(int u, int v)`
    - 辺 $uv$ を追加する
    - 時間計算量: $\mathrm{amortized}\ O(\log n)$
- `void cut(int v)`
    - 頂点 $v$ とその親を結ぶ辺を削除する
    - 時間計算量: $\mathrm{amortized}\ O(\log n)$
- `void evert(int v)`
    - 頂点 $v$ を木の根にする
    - 時間計算量: $\mathrm{amortized}\ O(\log n)$
- `void get(int v)`
    - 頂点 $v$ の値を取得する
    - 時間計算量: $O(1)$
- `void set(int v, T x)`
    - 頂点 $v$ の値を $x$ に変更する
    - 時間計算量: $\mathrm{amortized}\ O(\log n)$
- `T fold(int u, int v)`
    - $uv$ パス上の頂点の値を fold する
    - 時間計算量: $\mathrm{amortized}\ O(\log n)$
\end{markdown}
\end{small}

\begin{lstlisting}
template <typename M, typename M::T (*flip)(typename M::T)>
class LinkCutTree {
    using T = typename M::T;

public:
    LinkCutTree() = default;
    explicit LinkCutTree(int n) {
        for (int i = 0; i < n; ++i) {
            vertex.push_back(std::make_shared<Node>(M::id));
        }
    }

    void link(int v, int p) {
        evert(v);
        expose(vertex[p]);
        vertex[v]->par = vertex[p];
        vertex[p]->right = vertex[v];
        recalc(vertex[p]);
    }

    void cut(int v) {
        expose(vertex[v]);
        auto p = vertex[v]->left;
        vertex[v]->left = p->par = nullptr;
        recalc(vertex[v]);
    }

    void evert(int v) {
        expose(vertex[v]);
        reverse(vertex[v]);
    }

    T get(int v) const {
        return vertex[v]->val;
    }

    void set(int v, const T& x) {
        expose(vertex[v]);
        vertex[v]->val = x;
        recalc(vertex[v]);
    }

    T fold(int u, int v) {
        evert(u);
        expose(vertex[v]);
        return vertex[v]->sum;
    }

private:
    struct Node;
    using node_ptr = std::shared_ptr<Node>;

    struct Node {
        node_ptr left, right, par;
        T val, sum;
        int sz;
        bool rev;

        Node(const T& x)
            : left(nullptr), right(nullptr), par(nullptr),
              val(x), sum(x), sz(1), rev(false) {}
    };

    std::vector<node_ptr> vertex;

    static void expose(node_ptr v) {
        node_ptr prev = nullptr;
        for (auto cur = v; cur; cur = cur->par) {
            splay(cur);
            cur->right = prev;
            recalc(cur);
            prev = cur;
        }
        splay(v);
    }

    // splay tree

    static int size(const node_ptr& t) {
        return t ? t->sz : 0;
    }

    static void recalc(const node_ptr& t) {
        if (!t) return;
        t->sz = size(t->left) + 1 + size(t->right);
        t->sum = t->val;
        if (t->left) t->sum = M::op(t->left->sum, t->sum);
        if (t->right) t->sum = M::op(t->sum, t->right->sum);
    }

    static void push(const node_ptr& t) {
        if (t->rev) {
            if (t->left) reverse(t->left);
            if (t->right) reverse(t->right);
            t->rev = false;
        }
    }

    static void reverse(const node_ptr& t) {
        std::swap(t->left, t->right);
        t->sum = flip(t->sum);
        t->rev ^= true;
    }

    static void rotate_left(node_ptr t) {
        node_ptr s = t->right;
        t->right = s->left;
        if (s->left) s->left->par = t;
        s->par = t->par;
        if (t->par) {
            if (t->par->left == t) {
                t->par->left = s;
            }
            if (t->par->right == t) {
                t->par->right = s;
            }
        }
        s->left = t;
        t->par = s;
        recalc(t);
        recalc(s);
    }

    static void rotate_right(node_ptr t) {
        node_ptr s = t->left;
        t->left = s->right;
        if (s->right) s->right->par = t;
        s->par = t->par;
        if (t->par) {
            if (t->par->left == t) {
                t->par->left = s;
            }
            if (t->par->right == t) {
                t->par->right = s;
            }
        }
        s->right = t;
        t->par = s;
        recalc(t);
        recalc(s);
    }

    static bool is_root(const node_ptr& t) {
        return !t->par || (t->par->left != t && t->par->right != t);
    }

    static void splay(node_ptr t) {
        push(t);
        while (!is_root(t)) {
            auto p = t->par;
            if (is_root(p)) {
                push(p);
                push(t);
                if (t == p->left) rotate_right(p);
                else rotate_left(p);
            } else {
                auto g = p->par;
                push(g);
                push(p);
                push(t);
                if (t == p->left) {
                    if (p == g->left) {
                        rotate_right(g);
                        rotate_right(p);
                    } else {
                        rotate_right(p);
                        rotate_left(g);
                    }
                } else {
                    if (p == g->left) {
                        rotate_left(p);
                        rotate_right(g);
                    } else {
                        rotate_left(g);
                        rotate_left(p);
                    }
                }
            }
        }
    }
};

\end{lstlisting}
\section{math}

\subsection{Bostan-Mori Algorithm}

\begin{small}
\begin{markdown}
**Description**

Bostan-Mori algorithm は,$d$ 階線形漸化式の第 $n$ 項を高速に求めるアルゴリズムである.

**Operations**

- `T bostan\_mori\_division(Polynomial<T> p, Polynomial<T> q, long long n)`
    - $p(x)/q(x)$ の第 $n$ 項を求める.
    - 時間計算量: $O(\mathsf{M}(k) \log n)$, $\mathsf{M(k)}$ は$k$次多項式乗算の計算量 (FFT なら $O(k\log k)$)
- `T bostan\_mori\_recurrence(vector<T> a, vector<T> c, long long n)`
    - 初めの $k$ 項 $a\_0, \dots, a\_{k-1}$ と漸化式 $a\_n = c\_0 a\_{n-1} + \dots + c\_{k-1} a\_{n-k}$ によって定まる数列 $(a\_n)$ の第 $n$ 項を求める.
    - 時間計算量: 同上

\end{markdown}
\end{small}

\begin{lstlisting}
#include "../convolution/ntt.hpp"
#include "../math/polynomial.cpp"

template <typename T>
T bostan_mori_division(Polynomial<T> p, Polynomial<T> q, long long n) {
    auto take = [&](const std::vector<T>& p, int s) {
        std::vector<T> r((p.size() + 1) / 2);
        for (int i = 0; i < (int) r.size(); ++i) {
            if (2 * i + s < (int) p.size()) {
                r[i] = p[2 * i + s];
            }
        }
        return r;
    };

    while (n > 0) {
        auto qm = q;
        for (int i = 1; i < (int) qm.size(); i += 2) qm[i] = -qm[i];
        p = take(p * qm, n & 1);
        q = take(q * qm, 0);
        n >>= 1;
    }

    return p[0] / q[0];
}

template <typename T>
T bostan_mori_recurrence(const std::vector<T>& a, const std::vector<T>& c, long long n) {
    const int d = c.size();
    if (n < d) return a[n];

    std::vector<T> q(d + 1);
    q[0] = 1;
    for (int i = 0; i < d; ++i) q[i + 1] = -c[i];
    auto p = convolution(a, q);
    p.resize(d);

    auto take = [&](const std::vector<T>& p, int s) {
        std::vector<T> r((p.size() + 1) / 2);
        for (int i = 0; i < (int) r.size(); ++i) {
            r[i] = p[2 * i + s];
        }
        return r;
    };

    while (n > 0) {
        auto qm = q;
        for (int i = 1; i < (int) qm.size(); i += 2) qm[i] = -qm[i];
        p = take(convolution(p, qm), n & 1);
        q = take(convolution(q, qm), 0);
        n >>= 1;
    }

    return p[0] / q[0];
}
\end{lstlisting}
\section{misc}

\subsection{Mo's Algorithm}

\begin{small}
\begin{markdown}
- `Mo(int n)`
    - 長さ $n$ の列に対するクエリを処理する
    - 時間計算量: $O(1)$
- `void query(int l, int r)`
    - 区間 $[l, r)$ に対してクエリをする
    - 時間計算量: $O(1)$
- `void run(ExL exl, ShL shl, ExR exr, ShR shr, Out out)`
    - 以下の関数を引数に取り,クエリを実行する
        - `exl`: 区間を左に1マス伸ばしたときの状態を更新する
        - `shl`: 区間を左に1マス縮めたときの状態を更新する
        - `exr`: 区間を右に1マス伸ばしたときの状態を更新する
        - `shr`: 区間を右に1マス縮めたときの状態を更新する
        - `out`: $i$ 番目のクエリの結果を計算する
    - 時間計算量: $O(f(n)n\sqrt{n})$, $f(n)$ は状態の更新にかかる計算量
\end{markdown}
\end{small}

\begin{lstlisting}
class Mo {
public:
    Mo() = default;
    explicit Mo(int n) : n(n), cnt(0) {}

    void query(int l, int r) {
        queries.emplace_back(cnt++, l, r);
    }

    template <typename ExL, typename ShL, typename ExR, typename ShR, typename Out>
    void run(ExL exl, ShL shl, ExR exr, ShR shr, Out out) {
        int s = sqrt(n);
        std::sort(queries.begin(), queries.end(), [&](const auto& a, const auto& b) {
            if (a.l / s != b.l / s) return a.l < b.l;
            return a.r < b.r;
        });
        int curL = 0, curR = 0;
        for (auto [id, l, r] : queries) {
            while (curL > l) exl(--curL);
            while (curR < r) exr(curR++);
            while (curL < l) shl(curL++);
            while (curR > r) shr(--curR);
            out(id);
        }
    }

private:
    struct Query {
        int id, l, r;
        Query(int id, int l, int r) : id(id), l(l), r(r) {}
    };

    int n, cnt;
    std::vector<Query> queries;
};
\end{lstlisting}

\subsection{Interval Set}

\begin{small}
\begin{markdown}
**Description**

整数の閉区間の集合を管理する.

**Operations**

- `bool covered(T x)`
- `bool covered(T l, T r)`
    - 区間 $[l, r]$ が含まれているか判定する
    - 時間計算量: $O(\log n)$
- `pair<T, T> covered\_by(T x)`
- `pair<T, T> covered\_by(T l, T r)`
    - 区間 $[l, r]$ を含む区間を返す.そのような区間がない場合は $(-\infty, \infty)$ を返す
    - 時間計算量: $O(\log n)$
- `void insert(T x)`
- `void insert(T l, T r)`
    - 区間 $[l, r]$ を集合に追加する
    - 時間計算量: $\mathrm{amortized}\ O(\log n)$
- `void erase(T x)`
- `void erase(T l, T r)`
    - 区間 $[l, r]$ を集合から削除する
    - 時間計算量: $\mathrm{amortized}\ O(\log n)$
- `T mex(T x)`
    - $x$ 以上の整数のうち,集合に含まれない最小のものを返す
    - 時間計算量: $O(\log n)$

\end{markdown}
\end{small}

\begin{lstlisting}
template <typename T>
class IntervalSet {
public:
    static constexpr T INF = std::numeric_limits<T>::max() / 2;

    IntervalSet() {
        st.emplace(INF, INF);
        st.emplace(-INF, -INF);
    }

    bool covered(T x) const { return covered(x, x); }
    bool covered(T l, T r) const {
        assert(l <= r);
        auto it = --(st.lower_bound({l + 1, l + 1}));
        return it->first <= l && r <= it->second;
    }

    std::pair<T, T> covered_by(T x) const { return covered_by(x, x); }
    std::pair<T, T> covered_by(T l, T r) const {
        assert(l <= r);
        auto it = --(st.lower_bound({l + 1, l + 1}));
        if (it->first <= l && r <= it->second) return *it;
        return {-INF, -INF};
    }

    void insert(T x) { insert(x, x); }
    void insert(T l, T r) {
        assert(l <= r);
        auto it = --(st.lower_bound({l + 1, l + 1}));
        if (it->first <= l && r <= it->second) return;
        if (it->first <= l && l <= it->second + 1) {
            l = it->first;
            it = st.erase(it);
        } else {
            ++it;
        }
        while (it->second < r) {
            it = st.erase(it);
        }
        if (it->first - 1 <= r && r <= it->second) {
            r = it->second;
            st.erase(it);
        }
        st.emplace(l, r);
    }

    void erase(T x) { erase(x, x); }
    void erase(T l, T r) {
        assert(l <= r);
        auto it = --(st.lower_bound({l + 1, l + 1}));
        if (it->first <= l && r <= it->second) {
            if (it->first < l) st.emplace(it->first, l - 1);
            if (r < it->second) st.emplace(r + 1, it->second);
            st.erase(it);
            return;
        }
        if (it->first <= l && l <= it->second) {
            if (it->first < l) st.emplace(it->first, l - 1);
            it = st.erase(it);
        } else {
            ++it;
        }
        while (it->second <= r) {
            it = st.erase(it);
        }
        if (it->first <= r && r <= it->second) {
            if (r < it->second) st.emplace(r + 1, it->second);
            st.erase(it);
        }
    }

    std::set<std::pair<T, T>> ranges() const { return st; }

    T mex(T x) const {
        auto it = --(st.lower_bound({x + 1, x + 1}));
        if (it->first <= x && x <= it->second) return it->second + 1;
        return x;
    }

private:
    std::set<std::pair<T, T>> st;
};
\end{lstlisting}